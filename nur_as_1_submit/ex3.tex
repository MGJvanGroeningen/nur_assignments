\section{Satellite galaxies around a massive central}

In this question we look at the radial distribution of galaxies and the Poisson distribution.

\lstinputlisting{ex3_module.py}

\subsection{3a}

We calculate the normalization $A$ in
\begin{equation}
n(x)=A\left\langle N_{\text {sat }}\right\rangle\left(\frac{x}{b}\right)^{a-3} \exp \left[-\left(\frac{x}{b}\right)^{c}\right]
\end{equation}

with 

\begin{equation}
\iiint_{V} n(x) \mathrm{d} V=\left\langle N_{\mathrm{sat}}\right\rangle
\end{equation}

with boundaries $x_{min} = 10^{-4}$ and $x_{max} = 5$.

\lstinputlisting{ex3a.py}

Normalization $A$ is given by:

\lstinputlisting{3a_output.txt}

\subsection{3b}

\lstinputlisting{ex3b.py}

Here we interpolate 5 sample points from $n(x)$. I used linear interpolation and an akima spline. I did not have my own code for an akima spline so I used the scipy library. I realize that the idea was to not use code from special libraries, but the akima interpolation is a better fit to the true function. The main idea here is to do the interpolation in log space, as linear/polynomial interpolation of exponential functions in linear space tends to give bad results. 

\begin{figure}[h!]
  \centering
  \includegraphics[width=0.9\linewidth]{./3b_plot.pdf}
  \caption{Interpolation of 5 sample points in log space in the radial distribution of galaxy satellites. Two interpolators were used: linear (blue) and akima (orange).}
  \label{fig:3b}
\end{figure}

\newpage

\subsection{3c}

\lstinputlisting{ex3c.py}

Poisson distribution for a number of $\lambda$ and $k$.

\lstinputlisting{3c_output.txt}


