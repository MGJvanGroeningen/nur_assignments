\section{Satellite galaxies around a massive central}

The shared module for this exercise 

\lstinputlisting{ex2_module.py}

\subsection{2a}

\lstinputlisting{ex2a.py}

In order to find the maximum of $N(x)$ we find the minimum of $-N(x)$. We do this by first creating a bracket which contains the minimum and then tightening the bracket with a golden section search. The $x$ value for the maximum and the maximum of $N(x)$ are given respectively in the following output.

\lstinputlisting{nur_a2_2a.txt}

\subsection{2b}

\lstinputlisting{ex2b.py}

We create a distribution of galaxies according to the satellite density profile by using the rejection sample method. The distribution is saved for later questions. We then create a histogram of this distribution with logarithmically spaced bins and divide each bin by its width to weight the bins. We also overplot the histogram with the distribution function (see Fig. \ref{fig:2b}).

\newpage

\begin{figure}[!ht]
  \centering
  \includegraphics[width=0.9\linewidth]{./nur_a2_2b.pdf}
  \caption{Probability distribution function for satellites at a certain radius. A histogram is plotted of a particle distribution generated by the probability function.}
  \label{fig:2b}
\end{figure}

\newpage

\subsection{2c}

\lstinputlisting{ex2c.py}

For this question we use the galaxy distribution from the 2b. We take a 100 galaxy sample from this disribution, sort them and plot the number of galaxies enclosed in radius $r$.

\begin{figure}[!ht]
  \centering
  \includegraphics[width=0.9\linewidth]{./nur_a2_2c.pdf}
  \caption{The number of galaxies within radius $r$ for a 100 galaxy sample from a the satellite probability distribution.}
  \label{fig:2c}
\end{figure}

\newpage

\subsection{2d}

\lstinputlisting{ex2d.py}

We again take the galaxy distribution from 2b. We create 20 logarithmically space bins between 0.0001 and 5 kpc and find the bin which contains the most galaxies from the distribution of 2b. Then we create a sorted array of the galaxies in this most filled bin. The 16th percentile, median and 84th percentile are given respectively in the output below.

\lstinputlisting{nur_a2_2d.txt}

Next, we divide the galaxy distribution among 100 haloes, each containing 100 galaxies. Then we calculate for each halo how many of their galaxies are located in the bin containing most galaxies. We show this in a histogram where for every number between 0 and 100, the amount of haloes that contained that number of galaxies in the most filled bin is given (see Fig. \ref{fig:2d}). We also plot a Poisson distribution to compare with.

\begin{figure}[!ht]
  \centering
  \includegraphics[width=0.9\linewidth]{./nur_a2_2d.pdf}
  \caption{The amount of haloes that contain certain number of galaxies in the most filled bin. A Poisson distribution is also plotted with the mean of the number of galaxies in the most filled bin as the expectation value.}
  \label{fig:2d}
\end{figure}

